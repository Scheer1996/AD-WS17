%%%%%%%%%%%%%%%%%%%%%%%%%%%%%%%%%%%%%%%%%
% Short Sectioned Assignment
% LaTeX Template
% Version 1.0 (5/5/12)
%
% This template has been downloaded from:
% http://www.LaTeXTemplates.com
%
% Original author:
% Frits Wenneker (http://www.howtotex.com)
%
% License:
% CC BY-NC-SA 3.0 (http://creativecommons.org/licenses/by-nc-sa/3.0/)
%
%%%%%%%%%%%%%%%%%%%%%%%%%%%%%%%%%%%%%%%%%

%----------------------------------------------------------------------------------------
%	PACKAGES AND OTHER DOCUMENT CONFIGURATIONS
%----------------------------------------------------------------------------------------

\documentclass[paper=a4, fontsize=11pt]{scrartcl} % A4 paper and 11pt font size

\usepackage[T1]{fontenc} % Use 8-bit encoding that has 256 glyphs
\usepackage{fourier} % Use the Adobe Utopia font for the document - comment this line to return to the LaTeX default
\usepackage[english]{babel} % English language/hyphenation
\usepackage{amsmath,amsfonts,amsthm} % Math packages

\usepackage{lipsum} % Used for inserting dummy 'Lorem ipsum' text into the template

\usepackage{sectsty} % Allows customizing section commands
\allsectionsfont{\centering \normalfont\scshape} % Make all sections centered, the default font and small caps

\usepackage{fancyhdr} % Custom headers and footers
\pagestyle{fancyplain} % Makes all pages in the document conform to the custom headers and footers
\fancyhead{} % No page header - if you want one, create it in the same way as the footers below
\fancyfoot[L]{} % Empty left footer
\fancyfoot[C]{} % Empty center footer
\fancyfoot[R]{\thepage} % Page numbering for right footer
\renewcommand{\headrulewidth}{0pt} % Remove header underlines
\renewcommand{\footrulewidth}{0pt} % Remove footer underlines
\setlength{\headheight}{13.6pt} % Customize the height of the header

\numberwithin{equation}{section} % Number equations within sections (i.e. 1.1, 1.2, 2.1, 2.2 instead of 1, 2, 3, 4)
\numberwithin{figure}{section} % Number figures within sections (i.e. 1.1, 1.2, 2.1, 2.2 instead of 1, 2, 3, 4)
\numberwithin{table}{section} % Number tables within sections (i.e. 1.1, 1.2, 2.1, 2.2 instead of 1, 2, 3, 4)

\setlength\parindent{0pt} % Removes all indentation from paragraphs - comment this line for an assignment with lots of text

%----------------------------------------------------------------------------------------
%	TITLE SECTION
%----------------------------------------------------------------------------------------

\newcommand{\horrule}[1]{\rule{\linewidth}{#1}} % Create horizontal rule command with 1 argument of height
\usepackage[utf8]{inputenc}
\title{	
\normalfont \normalsize 
\textsc{Hochschule f\"ur Angewandte Wissenschaften Hamburg, Department Informatik} \\ [25pt] % Your university, school and/or department name(s)
\horrule{0.5pt} \\[0.4cm] % Thin top horizontal rule
\huge Effizienzvergleich drei verschiedener Implementationen des abstrakten Datentyps SET \\ % The assignment title
\horrule{2pt} \\[0.5cm] % Thick bottom horizontal rule
}

\author{Stefan Subotin, Paul Mathia , Dennis Sentler & Philip Scheer} % Your name

\date{\normalsize\today} % Today's date or a custom date

\begin{document}

\maketitle % Print the title
%----------------------------------------------------------------------------------------
%	PROBLEM 1
%----------------------------------------------------------------------------------------
Getestet wurden drei Implentationen des abstrakten Datentyps Set. Im folgenden betrachten wir die Implentationen ArrayList, DoubleLinkedSet und HeapList, die über eine Schnittstelle Set,
auf Effizienz sowie Vor- und Nachteile getestet wurden. Durch Verifikations- und Quantitative Tests werden diese drei Implementationen gegeneinander verglichen und auf Effizienz analysiert.\newline


\section{Dokumentation der Tests}
In diesem Abschnitt beschäftigen wir uns mit den Verifikationstests die bekräftigen sollen, das die Spezifikation aller drei Implementationen erfüllt ist. 

\subsection{testeSize}
Der erste zu testende Teil war die Implementation der size() Methode. Die size() Methode liefert den aktuellen Zähler über die bereits enthaltenen Elemente.
Im Test wurde also überprüft, ob das Hinzufügen sowie entfernen von Datenelementen den eigentlich beabsichtigten Effekt auf den Element-Zähler 'size' hatte. 
Getestet wurde das Hinzufügen von 10 Elementen in ein anfänglich mit 10 initialisiertes Array. Die Menge wurde anschließend mit 1000 Elementen befüllt
und es konnten keine Laufzeitspezifischen Fehler beobachtet werden. 

\subsection{testeAddAndFind}
Im folgenden werden die Methoden add() und find() auf Erfüllung der Spezifikation getestet. Die Methode add() fügt der Menge ein neues Element hinzu und liefert nach
Erfolgreicher Speicherung das POS-Objekt mit entsprechendem Index des hinzugefügten Elements. Die find() Methode erwartet als Parameter ein Objekt vom Typ Key, der die
Eigenschaft besitzt Elemente in der Menge eindeutig anhand ihres Schlüssels(KEY) zu identifizieren. Getestet wurde erst einmal ob Elemente die der Menge tatsächlich hinzugefügt
wurden, auch gefunden werden. In sämtlichen positiven Tests wurde die korrekte POS der hinzugefügten Element zurückgeliefert. Bei den negativ Tests wurde mit Hilfe eines
Stopelements überprüft ob das gesuchte Element auch tatsächlich nicht Teil der Menge ist.  

\subsection{testeAddAndDeletePos}
In diesem Teilabschnitt wurden weitere Tests auf die bereits eingeführte Methode add() in Kombination mit deletePos() durchgeführt.Die Methode deletePos() verlangt als
Parameter ein Pos-Objekt und entfernt das Element an dem vom Dienstleister übergebenen POS Parameter. Getestet wurden sowohl valide Positionen als auch Positionen die gar
nicht im Bereich der Listen Kapazität lagen. 

\subsection{testeAddAndDeleteKey}
Hier erfolgten weitere Tests zum Entfernen von Elementen diesmal allerdings mit der Methode deleteKey, welche als Parameter ein zu einem Element eindeutigen Schlüssel erwartet,
und falls vorhanden dieses entfernt. Getestet wurden auch hier valide sowie invalide Positionen getestet und es konnten keine Fehler über JUnit Tests beobachtet werden. 

\subsection{testeAddAndRetrieve}
Die Methode retrieve() liefert - angenommen die im Parameter übergebene Pos ist gültig - das Element an der jeweiligen Pos. Die Methode retrieve() arbeitet in Ihren Grundzügen 
ähnlich wie die anfänglich beschriebene Methode find(). Auch in diesem Test auf Spezifikation wurden alle zuvor hinzugefügten Elemente wiedergefunden. Der Negativfall lief genauso unproblematisch und 
ohne spezielle oder unerwartete Vorkommnisse ab. 

\subsection{testeUnify}
Die Vereinigungsmenge einer (nicht notwendigerweise nichtleeren) Menge U von Mengen ist die Menge der Objekte, die in mindestens einem Element von U enthalten sind.
Die mathematische Richtigkeit wurde in unserem Fall eingehalten, sodass zwei Mengen die über die unify() Methode vereinigt wurden nach ihrer Vereinigung keine Duplikate enthielten und die size()Methode 
den richtigen Wert der neu erzeugten Menge lieferte.  

\section{Auswertung Quantitative Tests}



\end{document}